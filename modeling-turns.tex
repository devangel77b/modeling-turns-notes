\documentclass{amsart}

% For revision control
\usepackage{rcs-multi}
\rcsid{$Id$}
\rcsid{$Header$}
\rcskwsave{$Author$}
\rcskwsave{$Date$} 
\rcskwsave{$Revision$}
%%\rcsRegisterAuthor{devangel}{Dennis Jos{\'e} Evangelista}
\rcsRegisterAuthor{devangel}{Dennis J. Evangelista}

\usepackage{graphicx}
\usepackage[usenames,dvipsnames]{color}
%\usepackage{makeidx} % incompatible with ams art
\usepackage{siunitx}
\DeclareMathOperator*{\argmin}{\arg\!\min}
\usepackage{multirow}
\usepackage{colortbl}

% PDF metadata
\usepackage{hyperref}
\hypersetup{pdftitle={Modeling multi-axis zero angular momentum turns the easy way}}
\hypersetup{pdfauthor={Rachel Becker, Dennis Evangelista, and Eliza McDonald}}
\hypersetup{pdfsubject={biology}}
\hypersetup{pdfkeywords={biomechanics, estimation, maneuverability, modeling, angular momentum, turns}}
\hypersetup{colorlinks=true,citecolor=Violet,linkcolor=Blue,urlcolor=Red}


\title{Modeling multi-axis zero angular momentum turns the easy way?}
\author{Rachel Becker, Dennis Evangelista, and Eliza McDonald}
\address{Department of Integrative Biology, UC Berkeley}
\email{devangel@berkeley.edu}
\thanks{Yu Zeng provided valuable discussions on this topic.}
\date{\today}

\begin{document}
\begin{abstract}
These notes give our thoughts on how to test a recorded movement for use of zero angular momentum turning mechanics. 
\end{abstract}
\maketitle
\tableofcontents

\section{Introduction}
This is an example of a citation \cite{Allain:2010, Akaike:1974, R:2011}.

We could also have an equation here:
\begin{equation}
H = J \omega
\label{eq:angmomentum}
\end{equation}
where $H$ might be angular momentum, $J$ is an inertia tensor, and $\omega$ is a vector of the angular velocities about the three principal axes.  It is possible to cite Equation~\ref{eq:angmomentum} fairly easily using labels and references in the \LaTeX code. Also note that it automagically makes neat color-coded hyper links to cross referenced stuff; that's really cool. 

This is an example of a figure:
\begin{figure}
\caption{Hello world.}
\label{fig:hello}
\end{figure}

This is an example of a table:
\begin{table}
\caption{Hello world in table form.}
\label{tab:hello}
\begin{tabular}{lr}
Able & Baker \\
Charle & Dog \\
\end{tabular}
\end{table}




\section{Methods and materials}
\section{Results}
\section{Discussion}

% AMS style references
\bibliographystyle{amsplain}
\bibliography{references/modeling-turns}
\end{document}
